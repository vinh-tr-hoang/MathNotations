\usepackage{amsmath}
%The principal package in the AMS-LATEX distribution. It adapts for use in LATEX most of the mathematical features found in AMS-TEX; it is highly recommended as an adjunct to serious mathematical typesetting in LATEX.
% https://ctan.org/pkg/amsmath?lang=en
\usepackage{amsfonts} 
% for \mathbb command
\usepackage{bm}
%The bm package defines a command \bm which makes its argument bold. The argument may be any maths object from a single symbol to an expression. This is closely related to the specification of the \boldsymbol command in AMS-LATEX, but \bm is rather more careful in the way it does things.
\usepackage{lmodern}
%  By default, LaTeX uses the Computer Modern font-harmony. Its sans-serif font-family doesn't have a bold + {italic/slanted} shape version. So you cannot get what you want.
% However, when you load e.g. the lmodern-package, you tell LaTeX to use the Latin Modern font-harmony.
% source: https://tex.stackexchange.com/questions/302597/how-to-write-bold-italic-and-sans-serif-at-the-same-time

%#### SCALAR, VECTOR, MATRIX, TENSOR
%scalar:  serifs, non-bold
%vector: serifs, non-bold
\newcommand{\vect}[1]{\bm{#1}}
\newcommand{\valpha}{\bm{\alpha}}
\newcommand{\va}{\bm{a}}
\newcommand{\vq}{\bm{q}}
\newcommand{\vx}{\bm{x}}
\newcommand{\vxi}{\bm{\xi}}
\newcommand{\vy}{\bm{y}}
%matrix: serifs, bold, upper case\\
\newcommand{\mat}[1]{\bm{#1}}
\newcommand{\mA}{\bm{A}}
\newcommand{\mQ}{\bm{Q}}
\newcommand{\mX}{\bm{X}}
\newcommand{\mXi}{\bm{\varXi}}
\newcommand{\mY}{\bm{Y}}
%tensor:  sans-serifs, bold, italic, $\textsf{\textbf{\textsl{q}}}= q_i^{jk}, \textsf{\textbf{\textsl{f}}}= q_i^{jk}$
\newcommand{\tns}[1]{\textsf{\textbf{\textsl{#1}}}}
\newcommand{\tnsA}{\tns{A}}
\newcommand{\tnsQ}{\tns{Q}}
\newcommand{\tnsX}{\tns{X}}
\newcommand{\tnsXi}{\tns{\varXi}}
\newcommand{\tnsY}{\tns{Y}}
%random variable, vector, matrix: serifs, non-bold, capital letter\\
\newcommand{\rv}[1]{#1}
\newcommand{\rvQ}{Q}
\newcommand{\rvX}{X}
\newcommand{\rvXi}{\varXi}
\newcommand{\rvY}{Y}
% operators
\newcommand{\secondnorm}[1]{||#1||}
\newcommand{\tr}{\textnormal{tr}}

%#### SETS
\newcommand{\sC}{\mathbb{C}}
\newcommand{\sR}{\mathbb{R}}
\newcommand{\sZ}{\mathbb{Z}}
\newcommand{\sN}{\mathbb{N}}
